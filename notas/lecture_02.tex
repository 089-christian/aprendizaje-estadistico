\documentclass[11pt,reqno,twoside]{article}
%>>>>>>> RENAME CURRENT FILE TO MATCH LECTURE NUMBER
% E.g., "lecture_01.tex"

%>>>>>>> DO NOT EDIT MACRO FILE

% >>>> DO NOT EDIT THIS FILE
% If you *must* add or change a macro, please email fkoh@caltech.edu

%=================================================
% Basics
%=================================================

\usepackage{fixltx2e} % Makes \( \) equation style robust, among other
                      % things. Must be the first package.


% Makes ligatured fonts searchable and copyable in pdf readers
\usepackage{cmap} % Load before fontenc 

% Always include these font encodings in your document 
% unless you have a very good reason.
\usepackage[T1]{fontenc}
\usepackage[utf8]{inputenc}

\usepackage{verbatim}

%=====================================
% Look & feel
%=====================================

% Allows for manual space setting
\usepackage{setspace}

%=============
% Fonts
%=============

\usepackage{lmodern} % Improved version of computer modern
\usepackage[scale=0.88]{tgheros} % Helvetica clone for sans serif font


\newcommand\hmmax{2} % Default is 3.
\newcommand\bmmax{2} % Default is 4.

\usepackage{bm} % boldmath must be called after the package
\providecommand{\mathbold}[1]{\bm{#1}}

%=============
% AMS Packages and fonts
%=============
\usepackage{amsmath,amsbsy,amsgen,amscd,amsthm,amsfonts,amssymb} 

%=============
% Margins and paper size
%=============
\usepackage[centering,top=1.5in,bottom=1.2in,left=1.4in,right=1.4in]{geometry}

%=============
% Title setup
%=============
\usepackage{titling}
%\usepackage{nopageno}
\setlength{\droptitle}{-7.5em}

\pretitle{\noindent\rule{0.85\linewidth}{0.2mm}\par%
  \begin{raggedright}\LARGE\sffamily}
\posttitle{\par\end{raggedright}%
\noindent%
\rule{0.85\linewidth}{0.5mm}\par}

\preauthor{\noindent\vspace{0.5em}%
  \sffamily\begin{tabular}[t]{ll}}
  \postauthor{\end{tabular}\par\thispagestyle{plain}}

\predate{\noindent%
  \small\sffamily\itshape\begin{tabular}[t]{l}%
    ACM 256, Winter 2019 \\ %
    Dr.\ Franca Hoffmann \\ %
  }
  \postdate{
  \end{tabular}\par}

%=============
% Section headings
%=============
\usepackage[sf,bf,compact]{titlesec}

%=============
% Tables and lists
%=============
\usepackage{booktabs,longtable,tabu} % Nice tables
\setlength{\tabulinesep}{1mm}
\usepackage[font=small,margin=10pt,labelfont={sf,bf},labelsep={space}]{caption}


\usepackage{enumitem}
\setitemize{itemsep=0pt} 
\setenumerate{itemsep=0pt}
\setlist{labelindent=\parindent,%  % Recommended by enumitem package
  font=\sffamily}


%=============
% Hyperlink colors
%=============
\usepackage[usenames,dvipsnames]{xcolor}
\definecolor{dark-gray}{gray}{0.3}
\definecolor{dkgray}{rgb}{.4,.4,.4}
\definecolor{dkblue}{rgb}{0,0,.5}
\definecolor{medblue}{rgb}{0,0,.75}
\definecolor{rust}{rgb}{0.5,0.1,0.1}

\usepackage{url}
\usepackage[colorlinks=true]{hyperref}
\hypersetup{linkcolor=dkblue}    
\hypersetup{citecolor=rust}      
\hypersetup{urlcolor=rust}     

%=============
% Microtype
%=============
\usepackage[final]{microtype} 

%=============
% Theorems, etc.
%=============
\newtheoremstyle{myThm} % name
    {\topsep}                    % Space above
    {\topsep}                    % Space below
    {\itshape}                   % Body font
    {}                           % Indent amount
    {\sffamily\bfseries}                   % Theorem head font
    {.}                          % Punctuation after theorem head
    {.5em}                       % Space after theorem head
    {}  % Theorem head spec (can be left empty, meaning ‘normal’)

\newtheoremstyle{myRem} % name
    {\topsep}                    % Space above
    {\topsep}                    % Space below
    {}                   % Body font
    {}                           % Indent amount
    {\sffamily}                   % Theorem head font
    {.}                          % Punctuation after theorem head
    {.5em}                       % Space after theorem head
    {}  % Theorem head spec (can be left empty, meaning ‘normal’)

\newtheoremstyle{myDef} % name
    {\topsep}                    % Space above
    {\topsep}                    % Space below
    {}                   % Body font
    {}                           % Indent amount
    {\sffamily\bfseries}                   % Theorem head font
    {.}                          % Punctuation after theorem head
    {.5em}                       % Space after theorem head
    {}  % Theorem head spec (can be left empty, meaning ‘normal’)

\theoremstyle{myThm}
\newtheorem{theorem}{Theorem}[section]
\newtheorem{lemma}[theorem]{Lemma}
\newtheorem{proposition}[theorem]{Proposition}
\newtheorem{corollary}[theorem]{Corollary}
\newtheorem{fact}[theorem]{Fact}

\theoremstyle{myRem}
\newtheorem{remark}[theorem]{Remark}

\theoremstyle{myDef}
\newtheorem{definition}[theorem]{Definition}
\newtheorem{example}[theorem]{Example}

%=====================
% Header
%=====================
\usepackage{fancyhdr}
\usepackage{nopageno} % Gets rid of page number at the bottom
\fancyhf{} % Clear header style
\renewcommand{\headrulewidth}{0.5pt} % remove the header rule
\pagestyle{fancy}
\fancyhead[LE,RO]{\textsf{\small \thepage}}

\setlength{\headheight}{14pt}
%=====================
% Fix delimiters
%=====================

% Fixes \left and \right spacing issues. See discussion at
% http://tex.stackexchange.com/questions/2607/spacing-around-left-and-right
\let\originalleft\left
\let\originalright\right
\renewcommand{\left}{\mathopen{}\mathclose\bgroup\originalleft}
\renewcommand{\right}{\aftergroup\egroup\originalright}

%=================================================
% Math macros
%=================================================

%=============
% Generalities
%=============
\usepackage{mathtools}
\mathtoolsset{centercolon}  % Makes := typeset correctly for definitions

%%% Equation numbering
%\numberwithin{equation}{section} 

%%% Annotations
\newcommand{\notate}[1]{\textcolor{red}{\textbf{[#1]}}}

%==============
% Symbols
%==============
\let\oldphi\phi
\let\oldeps\epsilon
\let\oldemptyset\emptyset
\let\emptyset\varnothing

\renewcommand{\phi}{\varphi}
\renewcommand{\epsilon}{\varepsilon}
\newcommand{\eps}{\varepsilon}
\newcommand{\cl}{\mathrm{cl}}
\newcommand{\wto}{\rightharpoonup}
\newcommand{\wsto}{\overset{\ast}{\rightharpoonup}}
\newcommand{\wwto}{\overset{w}{\to}}
\newcommand{\wwsto}{\overset{w*}{\to}}

%==============
% Constants
%==============

% Set constants upright
\newcommand{\cnst}[1]{\mathrm{#1}}  
\newcommand{\econst}{\mathrm{e}}
\newcommand{\rd}{\mathrm{d}}
\newcommand{\dist}{\mathrm{dist}}

\newcommand{\zerovct}{\vct{0}} % Zero vector
\newcommand{\Id}{\mathbf{I}} % Identity matrix
\newcommand{\onemtx}{\bm{1}}
\newcommand{\zeromtx}{\bm{0}}

%==============
% Sets
%==============
\providecommand{\mathbbm}{\mathbb} % In case we don't load bbm

% Reals, complex, naturals, integers, field
\newcommand{\R}{\mathbbm{R}}
\newcommand{\C}{\mathbbm{C}}
\newcommand{\N}{\mathbbm{N}}
\newcommand{\Z}{\mathbbm{Z}}
\newcommand{\F}{\mathbbm{F}}

%==============
% Probability
%==============
\newcommand{\Prob}{\operatorname{\mathbbm{P}}}
\newcommand{\Expect}{\operatorname{\mathbb{E}}}

%==============
% Vectors and matrices 
%==============
\newcommand{\vct}[1]{\mathbold{#1}}
\newcommand{\mtx}[1]{\mathbold{#1}}

%=============
% Operators
%=============
\newcommand{\B}{\mathcal{B}}
\newcommand{\op}[1]{\mathbold{#1}}

 % "macro.tex" must be in the same folder

%>>>>>>> IF NEEDED, ADD A NEW FILE WITH YOUR OWN MACROS

% \input{lecture_01_macro.tex} % Name of supplemental macros should match lecture number

%>>>>>>> LECTURE NUMBER AND TITLE
\title{Clase 02:               % UPDATE LECTURE NUMBER
    Modelo Formal de Aprendizaje}	% UPDATE TITLE
% TIP:  Use "\\" to break the title into more than one line.

%>>>>>>> DATE OF LECTURE
\date{Enero 19, 2021} % Hard-code lecture date. Don't use "\today"

%>>>>>>> NAME OF SCRIBE(S)
\author{%
  Responsable:&
  Manuel García Garduño  % >>>>> SCRIBE NAME(S)
}

\begin{document}
\maketitle %  LEAVE HERE
% The command above causes the title to be displayed.

%>>>>> DELETE ALL CONTENT UNTIL "\end{document}"
% This is the body of your document.

\section{Marco Formal del Aprendizaje Estadístico}
\label{sec:introduction}
\begin{enumerate}
    \item Las entradas
    \begin{itemize}
        \item Conjunto de dominio $\mathcal{X} \subseteq \R^d, d < \infty$.
        \item Conjunto de etiquetas $\mathcal{Y}$, por ejemplo, los conjuntos $\{0,1\}, \{-1,1\}$.
        \item Conjunto de entrenamiento: $S = \{(x_{i},y_{i}) , i = 1,...,m\}$; $m < \infty$, en donde $(x_{i},y_{i})\in\mathcal{X}\times\mathcal{Y}$.
    \end{itemize}
    \item La regla de predicción: $h:\mathcal{X}\mapsto\mathcal{Y}$.
    \item Un algoritmo de aprendizaje $A$, en donde $A(S)$ es la hipótesis que el algoritmo de aprendizaje genera al observar el conjunto de entrenamiento.
    \item Un modelo que genera los datos
    \begin{itemize}
        \item[$i)$] Asumimos que $\mathcal{X}$ tiene una medida de probabilidad $\mathcal{D}$ (distribución) que se desconoce.
        \item[$ii)$] Asumimos que existe una función que etiqueta correctamente los datos, es decir, $\exists$ $f:\mathcal{X}\mapsto\mathcal{Y}$ tal que $f(x_{i}) = y_{i}$
    \end{itemize}
    \item Una métrica de éxito
    \begin{definition}[Error del clasificador]
    El error de un clasificador $h$, es la probabilidad de etiquetar incorrectamente una instancia generada por $\mathcal{D}$.
    \end{definition}
    \begin{itemize}
        \item El error de $h$ también puede expresarse como:
        \begin{equation}
            \mathcal{L}_{(\mathcal{D},f)}(h) = P\{h(x) \neq f(x))\} = \mathcal{D}(\{x:h(x)\neq f(x)\})
        \end{equation}
        \item Conocemos $S$, pero desconocemos $f$ y $\mathcal{D}$
    \end{itemize}
\end{enumerate}
\section{Minimización del Riesgo Empírico}
A pesar de no poder calcular el verdadero error de clasificación puesto que ignoramos a la función que etiqueta correctamente a los elementos de $\mathcal{X}$ y tampoco conocemos su distribución, sí podemos construir una medida del error que es calculable con los datos que tenemos.
\begin{definition}[Riesgo Empírico]
    Para un subconjunto de $m$ elementos de $\mathcal{X}$, definimos el riesgo empírico de nuestra regla de predicción $h$ como
    \begin{equation}
        L_{S}(h) = \frac{|\{i = 1,...,m : h(x_{i}) \neq f(x_{i})\}|}{m} \,.
    \end{equation}
\end{definition}

\subsection{¿Qué podría salir mal?}
Supongamos que a partir de nuestro conjunto de entrenamiento $S$ decidimos definir la siguiente regla de predicción:
\begin{equation}
h(x) =
    \begin{cases}
      y_{i} & \textrm{ si existe } i \textrm{ tal que } x_{i} = x\\
      0     & \textrm{ en otro caso}\\
    \end{cases}\,.
\end{equation}

Realmente lo que la regla de prediccón está haciendo es asignar a cada valor de
$x$ el valor de $y$ que se observa en el conjunto de entrenamiento, y si el
valor de $x$ no se encontraba en el conjunto de entrenamiento original le asigna
el valor $0$. Observemos que el error empírico de $h$ calculado sobre los datos
de entrenamiento es cero, puesto que la forma en que $h$ fue definida nos
garantiza que siempre va a etiquetar correctamente a los datos de entrenamiento.
Pero, ¿acaso $h$ etiqueta correctamente a las $x$'s que no estaban en el
conjunto de entrenamiento? Posiblemente no, porque $h$ siempre les asignará la
etiqueta $O$ . Más aún, el verdadero error del clasificador (que es el error que
verdaderamente nos importa) será muy grande. Moraleja: minimizar el riesgo
empírico no minimiza el error real del clasificador.

\subsection{ERM con sesgo inductivo}

Para solucionar este problema se escoge, antes de ver los datos, una familia
(clase) $\mathcal{H}$ de posibles candidatos para $h$. De esta forma, quizá no
se minimice el error empírico, pero el modelo tendrá una mayor capacidad de
generalización que reduzca el error real del clasificador.


%>>>>>> END OF YOUR CONTENT

\bibliographystyle{siam} % <<< USE "alpha" BIBLIOGRAPHY STYLE
\bibliography{template} % <<< RENAME TO "lecture_XX"


\end{document}
