\documentclass[11pt,reqno,twoside]{article}
%>>>>>>> RENAME CURRENT FILE TO MATCH LECTURE NUMBER
% E.g., "lecture_01.tex"

%>>>>>>> DO NOT EDIT MACRO FILE

% >>>> DO NOT EDIT THIS FILE
% If you *must* add or change a macro, please email fkoh@caltech.edu

%=================================================
% Basics
%=================================================

\usepackage{fixltx2e} % Makes \( \) equation style robust, among other
                      % things. Must be the first package.


% Makes ligatured fonts searchable and copyable in pdf readers
\usepackage{cmap} % Load before fontenc 

% Always include these font encodings in your document 
% unless you have a very good reason.
\usepackage[T1]{fontenc}
\usepackage[utf8]{inputenc}

\usepackage{verbatim}

%=====================================
% Look & feel
%=====================================

% Allows for manual space setting
\usepackage{setspace}

%=============
% Fonts
%=============

\usepackage{lmodern} % Improved version of computer modern
\usepackage[scale=0.88]{tgheros} % Helvetica clone for sans serif font


\newcommand\hmmax{2} % Default is 3.
\newcommand\bmmax{2} % Default is 4.

\usepackage{bm} % boldmath must be called after the package
\providecommand{\mathbold}[1]{\bm{#1}}

%=============
% AMS Packages and fonts
%=============
\usepackage{amsmath,amsbsy,amsgen,amscd,amsthm,amsfonts,amssymb} 

%=============
% Margins and paper size
%=============
\usepackage[centering,top=1.5in,bottom=1.2in,left=1.4in,right=1.4in]{geometry}

%=============
% Title setup
%=============
\usepackage{titling}
%\usepackage{nopageno}
\setlength{\droptitle}{-7.5em}

\pretitle{\noindent\rule{0.85\linewidth}{0.2mm}\par%
  \begin{raggedright}\LARGE\sffamily}
\posttitle{\par\end{raggedright}%
\noindent%
\rule{0.85\linewidth}{0.5mm}\par}

\preauthor{\noindent\vspace{0.5em}%
  \sffamily\begin{tabular}[t]{ll}}
  \postauthor{\end{tabular}\par\thispagestyle{plain}}

\predate{\noindent%
  \small\sffamily\itshape\begin{tabular}[t]{l}%
    ACM 256, Winter 2019 \\ %
    Dr.\ Franca Hoffmann \\ %
  }
  \postdate{
  \end{tabular}\par}

%=============
% Section headings
%=============
\usepackage[sf,bf,compact]{titlesec}

%=============
% Tables and lists
%=============
\usepackage{booktabs,longtable,tabu} % Nice tables
\setlength{\tabulinesep}{1mm}
\usepackage[font=small,margin=10pt,labelfont={sf,bf},labelsep={space}]{caption}


\usepackage{enumitem}
\setitemize{itemsep=0pt} 
\setenumerate{itemsep=0pt}
\setlist{labelindent=\parindent,%  % Recommended by enumitem package
  font=\sffamily}


%=============
% Hyperlink colors
%=============
\usepackage[usenames,dvipsnames]{xcolor}
\definecolor{dark-gray}{gray}{0.3}
\definecolor{dkgray}{rgb}{.4,.4,.4}
\definecolor{dkblue}{rgb}{0,0,.5}
\definecolor{medblue}{rgb}{0,0,.75}
\definecolor{rust}{rgb}{0.5,0.1,0.1}

\usepackage{url}
\usepackage[colorlinks=true]{hyperref}
\hypersetup{linkcolor=dkblue}    
\hypersetup{citecolor=rust}      
\hypersetup{urlcolor=rust}     

%=============
% Microtype
%=============
\usepackage[final]{microtype} 

%=============
% Theorems, etc.
%=============
\newtheoremstyle{myThm} % name
    {\topsep}                    % Space above
    {\topsep}                    % Space below
    {\itshape}                   % Body font
    {}                           % Indent amount
    {\sffamily\bfseries}                   % Theorem head font
    {.}                          % Punctuation after theorem head
    {.5em}                       % Space after theorem head
    {}  % Theorem head spec (can be left empty, meaning ‘normal’)

\newtheoremstyle{myRem} % name
    {\topsep}                    % Space above
    {\topsep}                    % Space below
    {}                   % Body font
    {}                           % Indent amount
    {\sffamily}                   % Theorem head font
    {.}                          % Punctuation after theorem head
    {.5em}                       % Space after theorem head
    {}  % Theorem head spec (can be left empty, meaning ‘normal’)

\newtheoremstyle{myDef} % name
    {\topsep}                    % Space above
    {\topsep}                    % Space below
    {}                   % Body font
    {}                           % Indent amount
    {\sffamily\bfseries}                   % Theorem head font
    {.}                          % Punctuation after theorem head
    {.5em}                       % Space after theorem head
    {}  % Theorem head spec (can be left empty, meaning ‘normal’)

\theoremstyle{myThm}
\newtheorem{theorem}{Theorem}[section]
\newtheorem{lemma}[theorem]{Lemma}
\newtheorem{proposition}[theorem]{Proposition}
\newtheorem{corollary}[theorem]{Corollary}
\newtheorem{fact}[theorem]{Fact}

\theoremstyle{myRem}
\newtheorem{remark}[theorem]{Remark}

\theoremstyle{myDef}
\newtheorem{definition}[theorem]{Definition}
\newtheorem{example}[theorem]{Example}

%=====================
% Header
%=====================
\usepackage{fancyhdr}
\usepackage{nopageno} % Gets rid of page number at the bottom
\fancyhf{} % Clear header style
\renewcommand{\headrulewidth}{0.5pt} % remove the header rule
\pagestyle{fancy}
\fancyhead[LE,RO]{\textsf{\small \thepage}}

\setlength{\headheight}{14pt}
%=====================
% Fix delimiters
%=====================

% Fixes \left and \right spacing issues. See discussion at
% http://tex.stackexchange.com/questions/2607/spacing-around-left-and-right
\let\originalleft\left
\let\originalright\right
\renewcommand{\left}{\mathopen{}\mathclose\bgroup\originalleft}
\renewcommand{\right}{\aftergroup\egroup\originalright}

%=================================================
% Math macros
%=================================================

%=============
% Generalities
%=============
\usepackage{mathtools}
\mathtoolsset{centercolon}  % Makes := typeset correctly for definitions

%%% Equation numbering
%\numberwithin{equation}{section} 

%%% Annotations
\newcommand{\notate}[1]{\textcolor{red}{\textbf{[#1]}}}

%==============
% Symbols
%==============
\let\oldphi\phi
\let\oldeps\epsilon
\let\oldemptyset\emptyset
\let\emptyset\varnothing

\renewcommand{\phi}{\varphi}
\renewcommand{\epsilon}{\varepsilon}
\newcommand{\eps}{\varepsilon}
\newcommand{\cl}{\mathrm{cl}}
\newcommand{\wto}{\rightharpoonup}
\newcommand{\wsto}{\overset{\ast}{\rightharpoonup}}
\newcommand{\wwto}{\overset{w}{\to}}
\newcommand{\wwsto}{\overset{w*}{\to}}

%==============
% Constants
%==============

% Set constants upright
\newcommand{\cnst}[1]{\mathrm{#1}}  
\newcommand{\econst}{\mathrm{e}}
\newcommand{\rd}{\mathrm{d}}
\newcommand{\dist}{\mathrm{dist}}

\newcommand{\zerovct}{\vct{0}} % Zero vector
\newcommand{\Id}{\mathbf{I}} % Identity matrix
\newcommand{\onemtx}{\bm{1}}
\newcommand{\zeromtx}{\bm{0}}

%==============
% Sets
%==============
\providecommand{\mathbbm}{\mathbb} % In case we don't load bbm

% Reals, complex, naturals, integers, field
\newcommand{\R}{\mathbbm{R}}
\newcommand{\C}{\mathbbm{C}}
\newcommand{\N}{\mathbbm{N}}
\newcommand{\Z}{\mathbbm{Z}}
\newcommand{\F}{\mathbbm{F}}

%==============
% Probability
%==============
\newcommand{\Prob}{\operatorname{\mathbbm{P}}}
\newcommand{\Expect}{\operatorname{\mathbb{E}}}

%==============
% Vectors and matrices 
%==============
\newcommand{\vct}[1]{\mathbold{#1}}
\newcommand{\mtx}[1]{\mathbold{#1}}

%=============
% Operators
%=============
\newcommand{\B}{\mathcal{B}}
\newcommand{\op}[1]{\mathbold{#1}}

 % "macro.tex" must be in the same folder

%>>>>>>> IF NEEDED, ADD A NEW FILE WITH YOUR OWN MACROS

% \input{lecture_01_macro.tex} % Name of supplemental macros should match lecture number

%>>>>>>> LECTURE NUMBER AND TITLE
\title{Clase 04:               % UPDATE LECTURE NUMBER
    El Modelo de Aprendizaje}	% UPDATE TITLE
% TIP:  Use "\\" to break the title into more than one line.

%>>>>>>> DATE OF LECTURE
\date{Enero 26, 2021} % Hard-code lecture date. Don't use "\today"

%>>>>>>> NAME OF SCRIBE(S)
\author{%
  Responsable:&
  Jordi Legorreta Montaño  % >>>>> SCRIBE NAME(S)
}

\begin{document}
\maketitle %  LEAVE HERE
% The command above causes the title to be displayed.

%>>>>> DELETE ALL CONTENT UNTIL "\end{document}"
% This is the body of your document.

\begin{itemize}
    \item Seguiremos considerando el modelo PAC (modelo Posible y Aproximadamente Correcto).
    \item Más adelante veremos otras nociones de aprendizaje.
    \item Pioneros en escribir este modelo formal de aprendizaje: Vapnik y Chervonenkis.
\end{itemize}

\section{PAC Aprendible o Aprendizaje PAC}
\label{sec:PAC}

$$\vert\mathcal{H}\vert<\infty\hspace{5mm}\oplus\hspace{5mm}ERM\hspace{5mm}\oplus\hspace{5mm}m\geq M_{0}$$

\begin{description}\setlength{\itemsep}{0pt} % Compact description list
    \item [Definición]: una clase $\mathcal{H}$ es aprendible, en el sentido PAC, si existe una función $m_{\mathcal{H}}:(0,1)^2\to\mathbb{N}$, y tenemos un algoritmo de aprendizaje con la siguiente propiedad:

    \begin{center}
        \begin{minipage}[c]{0.7\linewidth}
            $\forall\hspace{2mm}\epsilon,\delta\in(0,1)$\\
            $\forall\hspace{2mm}D$ sobre $\mathcal{X}$\\
            $f:\mathcal{X}\to\mathcal{Y}=\{0,1\}$
        \end{minipage}
    \end{center}
\end{description}

Si se satisface la hipótesis de realizabilidad con respecto a $\mathcal{H}$, D y $f$, entonces:
si utilizamos el algoritmo de aprendizaje con una cantidad suficiente de datos, $m\geq m_{\mathcal{H}}(\epsilon,\delta)$, el algoritmo regresa una $h_{s}$ tal que, con probabilidad $(1-\delta)$, satisface $L_{D,f}(h_{s})\leq \epsilon$.

\begin{center}
    \begin{minipage}[c]{0.7\linewidth}
        $\epsilon$ : precisión (qué tan lejos va a estar el clasificador con respecto a el óptimo).\\
        $\delta$ : parámetro de confianza (qué tan probable será que el clasificador que encontremos alcance el requerimiento de precisión).
    \end{minipage}
\end{center}

\subsection{¿Hay forma de evitar estas aproximaciones (en términos de $\epsilon$ y $\delta$)?}

No, no se pueden evitar, ya que no tenemos acceso a $D$, sino sólo a una muestra $S$, y no podemos garantizar que una muestra sea representativa.

$m\geq m_{\mathcal{H}}(\epsilon,\delta)$: complejidad muestral, es decir, el número mínimo de observaciones independientes de la distribución que necesitamos para poder establecer las garantías anteriores.

\begin{description}\setlength{\itemsep}{0pt} % Compact description list
    \item [Observación]: si $\mathcal{H}$ es PAC aprendible, existen muchas posibilidades para $m_{\mathcal{H}}$, por lo que acotamos $m_{\mathcal{H}}$ para que sea la mínima función en el sentido de que nos dará el mínimo entero necesario, considerando $\epsilon$ y $\delta$.
    \item [Corolario]: toda clase finita de hipótesis ($\mathcal{H}$) es PAC aprendible si:
    $$m_{\mathcal{H}}(\epsilon,\delta)\leq\Bigg\lceil \frac{log(\frac{\vert\mathcal{H}\vert}{\delta})}{\epsilon}\Bigg\rceil$$
    
\end{description}

\subsection{¿En qué sentido podemos generalizar PAC?}

\begin{itemize}
    \item ¿Sólo clasificación?, es decir, ¿sólo 0 ó 1?
    \item La hipótesis de realizabilidad podría no ser la correcta, en el sentido de que posiblemente no podríamos garantizar que un algoritmo de aprendizaje fuese exitoso para una distribución fija y una función, en cualquier otro contexto de datos.
\end{itemize}

\subsubsection{Aprendizaje PAC agnóstico}

Antes: $\exists h^*\in\mathcal{H}\hspace{2mm}\mid\hspace{2mm} \mathbb{P}_{\mathcal{X}\sim D}\Big(h^*(x)=f(x)\Big)=1$\\

Ahora:

\begin{center}
    \begin{minipage}[c]{0.7\linewidth}
        $D$ como una medida de probabilidad en el espacio $\mathcal{X}\times\mathcal{Y}$\\
        $D_{x}$ distribución marginal en $\mathcal{X}$\\
        $D_{y\mid x}$ distribución condicional para $\mathcal{Y}$
    \end{minipage}
\end{center}

Es decir, estamos ahora admitiendo la posibilidad de que dos elementos que tengan características muy similares (si no es que iguales) puedan tener una etiqueta distinta.

Por lo tanto, rompiendo con la hipótesis de realizabilidad, podemos definir una función de pérdida:

$$L_{D}(h)\stackrel{def}{=}\mathbb{P}_{(x,y)\sim D}\Big(h(x)\neq y\Big)\stackrel{def}{=}D\Big(\{(x,y):h(x)\neq y\}\Big)$$

$$L_{S}(h)=\frac{\vert\{i:h(x^{(i)})\neq y^{(i)}\vert}{m}, \hspace{8mm}(x^{(i)},y^{(i)})\stackrel{iid}{\sim}D$$\vspace{2mm}

Antes: la aleatoriedad provenía sólo de $\mathcal{X}$.\vspace{2.5mm}

Ahora: extendemos $D$ para que sea una medida de probabilidad en la pareja $(x,y)$.\vspace{2.5mm}

Se puede ver como: $D_{y\mid x}$ sustituye a $f(x)=y$, ya que la relción entre $\mathcal{X}$ e $\mathcal{Y}$ no es determinista, no es una función, sino que ahora es un modelo probabilístico.

\newpage

\begin{description}\setlength{\itemsep}{0pt} % Compact description list
    \item [Definición]: Clasificador Bayesiano Óptimo
        \begin{center}
            \begin{minipage}[c]{0.7\linewidth}
                Dada la distribución $D$ sobre $\mathcal{X}\times\{0,1\}$, el mejor predictor de etiquetas es:
                \[   
                f_{D}(x) = 
                    \begin{cases}
                        \text{1} \quad\text{si\quad $\mathbb{P}(y=1\mid x)\geq1/2$}\\
                        \text{0} \quad\text{si\quad $\mathbb{P}(y=1\mid x)<1/2$}
                    \end{cases}
                \]
                es óptimo porque cualquier otro clasificador tiene un error (riesgo) mayor que $f_{D}$.
                $$g:\mathcal{X}\to\{0,1\}, \quad\text{entonces}\quad L_{D}(f_{D})\leq L_{D}(g)$$
            \end{minipage}
        \end{center}
    \item [Definición]: Capacidad de aprendizaje PAC agnóstico
        \begin{center}
            \begin{minipage}[c]{0.7\linewidth}
                Una clase de hipótesis es PAC aprendible en un sentido agnóstico si existe una función de complejidad muestral $m_{\mathcal{H}}(0,1)^2\to\mathbb{N}$ y un algoritmo de aprendizaje con la siguiente propiedad:
                $$\forall(\epsilon,\delta)\in(0,1)\hspace{3mm}\text{y}\hspace{3mm}\forall D\hspace{2mm}\text{sobre}\hspace{2mm}\mathcal{X}\times{\{0,1\}}$$
                si el algoritmo de aprendizaje utiliza un número suficiente de observaciones $m\geq m_{\mathcal{H}}(\epsilon,\delta)$ $iid$ de $D$, entonces encontraremos una hipótesis $h$ con probabilidad $(1-\delta)$ que satisface:
                $$L_{D}(h)\leq \min_{h'\in\mathcal{H}}L_{D}(h')+\epsilon$$
            \end{minipage}
        \end{center}
\end{description}

\begin{itemize}
    \item Si no se satisface la hipótesis de realizabilidad, entonces no podemos garantizar que el error de generalización ($L_{D}(h)$) sea arbitrariamente chico.
    \item PAC agnóstico puede ser exitoso aún cuando no alcancemos el mejor error posible sobre la clase de hipótesis con la que estamos trabajando.
    \item La precisión de nuestra generalización se vuelve relativa a la clase con la que estamos trabajando, precisamente por la constante $\displaystyle\min_{h'\in\mathcal{H}}L_{D}(h')$
\end{itemize}

\subsubsection{Más allá de clasificación}

\begin{description}\setlength{\itemsep}{0pt} % Compact description list

    \item [Regresión]:
        \begin{center}
            \begin{minipage}[c]{0.7\linewidth}
                $y\in\mathbb{R}\quad\therefore\quad\mathcal{Y}=\mathbb{R}\hspace{2mm}\text{(recta real)}$.\\
                $l(x,y)=(h(x)-y)^2$, donde $l(x,y)$ es la función de pérdida.\\
                $L_{D}(h)=\mathbb{E}_{(x,y)\sim D}[l(x,y)]$, donde $L_{D}(h)$ es la función de riesgo.
            \end{minipage}
        \end{center}
\end{description}

Antes: conjunto de etiquetas $\mathcal{Y}$\vspace{2.5}

Ahora: conjunto objetivo $\mathcal{Y}$

\begin{description}\setlength{\itemsep}{0pt} % Compact description list

    \item [Funciones de pérdida]:
        \begin{center}
            \begin{minipage}[c]{0.7\linewidth}
                $\mathcal{H}$: clase de hipótesis, y el dominio $Z=\mathcal{X}\times\mathcal{Y}$\\
                $l:\mathcal{H}\times Z\to\mathbb{R}^{+}$\\
                $L_{D}(h)=\mathbb{E}_{Z\sim D}[l(h,z)]$\\
                $L_{S}(h)=\frac{1}{m}\sum_{i=1}^{m}l(h,z^{(i)})$\\
                $Z^{(i)}\stackrel{iid}{\sim}D$\\
                $L_{S}(h)$ es un estimador insesgado
            \end{minipage}
        \end{center}


%>>>>>> END OF YOUR CONTENT

%>>>>>>>>>> take acknowledgements out
\section*{Agradecimientos} Este {\emph template} se ha adaptado y traducido del
provisto en la clase ACM 204 (Otoño 2017) por el profesor Joel Tropp.


\bibliographystyle{siam} % <<< USE "alpha" BIBLIOGRAPHY STYLE
\bibliography{template} % <<< RENAME TO "lecture_XX"


\end{document}