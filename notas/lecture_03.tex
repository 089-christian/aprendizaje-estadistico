\documentclass[11pt,reqno,twoside]{article}
%>>>>>>> RENAME CURRENT FILE TO MATCH LECTURE NUMBER
% E.g., "lecture_01.tex"

%>>>>>>> DO NOT EDIT MACRO FILE

% >>>> DO NOT EDIT THIS FILE
% If you *must* add or change a macro, please email fkoh@caltech.edu

%=================================================
% Basics
%=================================================

\usepackage{fixltx2e} % Makes \( \) equation style robust, among other
                      % things. Must be the first package.


% Makes ligatured fonts searchable and copyable in pdf readers
\usepackage{cmap} % Load before fontenc 

% Always include these font encodings in your document 
% unless you have a very good reason.
\usepackage[T1]{fontenc}
\usepackage[utf8]{inputenc}

\usepackage{verbatim}

%=====================================
% Look & feel
%=====================================

% Allows for manual space setting
\usepackage{setspace}

%=============
% Fonts
%=============

\usepackage{lmodern} % Improved version of computer modern
\usepackage[scale=0.88]{tgheros} % Helvetica clone for sans serif font


\newcommand\hmmax{2} % Default is 3.
\newcommand\bmmax{2} % Default is 4.

\usepackage{bm} % boldmath must be called after the package
\providecommand{\mathbold}[1]{\bm{#1}}

%=============
% AMS Packages and fonts
%=============
\usepackage{amsmath,amsbsy,amsgen,amscd,amsthm,amsfonts,amssymb} 

%=============
% Margins and paper size
%=============
\usepackage[centering,top=1.5in,bottom=1.2in,left=1.4in,right=1.4in]{geometry}

%=============
% Title setup
%=============
\usepackage{titling}
%\usepackage{nopageno}
\setlength{\droptitle}{-7.5em}

\pretitle{\noindent\rule{0.85\linewidth}{0.2mm}\par%
  \begin{raggedright}\LARGE\sffamily}
\posttitle{\par\end{raggedright}%
\noindent%
\rule{0.85\linewidth}{0.5mm}\par}

\preauthor{\noindent\vspace{0.5em}%
  \sffamily\begin{tabular}[t]{ll}}
  \postauthor{\end{tabular}\par\thispagestyle{plain}}

\predate{\noindent%
  \small\sffamily\itshape\begin{tabular}[t]{l}%
    ACM 256, Winter 2019 \\ %
    Dr.\ Franca Hoffmann \\ %
  }
  \postdate{
  \end{tabular}\par}

%=============
% Section headings
%=============
\usepackage[sf,bf,compact]{titlesec}

%=============
% Tables and lists
%=============
\usepackage{booktabs,longtable,tabu} % Nice tables
\setlength{\tabulinesep}{1mm}
\usepackage[font=small,margin=10pt,labelfont={sf,bf},labelsep={space}]{caption}


\usepackage{enumitem}
\setitemize{itemsep=0pt} 
\setenumerate{itemsep=0pt}
\setlist{labelindent=\parindent,%  % Recommended by enumitem package
  font=\sffamily}


%=============
% Hyperlink colors
%=============
\usepackage[usenames,dvipsnames]{xcolor}
\definecolor{dark-gray}{gray}{0.3}
\definecolor{dkgray}{rgb}{.4,.4,.4}
\definecolor{dkblue}{rgb}{0,0,.5}
\definecolor{medblue}{rgb}{0,0,.75}
\definecolor{rust}{rgb}{0.5,0.1,0.1}

\usepackage{url}
\usepackage[colorlinks=true]{hyperref}
\hypersetup{linkcolor=dkblue}    
\hypersetup{citecolor=rust}      
\hypersetup{urlcolor=rust}     

%=============
% Microtype
%=============
\usepackage[final]{microtype} 

%=============
% Theorems, etc.
%=============
\newtheoremstyle{myThm} % name
    {\topsep}                    % Space above
    {\topsep}                    % Space below
    {\itshape}                   % Body font
    {}                           % Indent amount
    {\sffamily\bfseries}                   % Theorem head font
    {.}                          % Punctuation after theorem head
    {.5em}                       % Space after theorem head
    {}  % Theorem head spec (can be left empty, meaning ‘normal’)

\newtheoremstyle{myRem} % name
    {\topsep}                    % Space above
    {\topsep}                    % Space below
    {}                   % Body font
    {}                           % Indent amount
    {\sffamily}                   % Theorem head font
    {.}                          % Punctuation after theorem head
    {.5em}                       % Space after theorem head
    {}  % Theorem head spec (can be left empty, meaning ‘normal’)

\newtheoremstyle{myDef} % name
    {\topsep}                    % Space above
    {\topsep}                    % Space below
    {}                   % Body font
    {}                           % Indent amount
    {\sffamily\bfseries}                   % Theorem head font
    {.}                          % Punctuation after theorem head
    {.5em}                       % Space after theorem head
    {}  % Theorem head spec (can be left empty, meaning ‘normal’)

\theoremstyle{myThm}
\newtheorem{theorem}{Theorem}[section]
\newtheorem{lemma}[theorem]{Lemma}
\newtheorem{proposition}[theorem]{Proposition}
\newtheorem{corollary}[theorem]{Corollary}
\newtheorem{fact}[theorem]{Fact}

\theoremstyle{myRem}
\newtheorem{remark}[theorem]{Remark}

\theoremstyle{myDef}
\newtheorem{definition}[theorem]{Definition}
\newtheorem{example}[theorem]{Example}

%=====================
% Header
%=====================
\usepackage{fancyhdr}
\usepackage{nopageno} % Gets rid of page number at the bottom
\fancyhf{} % Clear header style
\renewcommand{\headrulewidth}{0.5pt} % remove the header rule
\pagestyle{fancy}
\fancyhead[LE,RO]{\textsf{\small \thepage}}

\setlength{\headheight}{14pt}
%=====================
% Fix delimiters
%=====================

% Fixes \left and \right spacing issues. See discussion at
% http://tex.stackexchange.com/questions/2607/spacing-around-left-and-right
\let\originalleft\left
\let\originalright\right
\renewcommand{\left}{\mathopen{}\mathclose\bgroup\originalleft}
\renewcommand{\right}{\aftergroup\egroup\originalright}

%=================================================
% Math macros
%=================================================

%=============
% Generalities
%=============
\usepackage{mathtools}
\mathtoolsset{centercolon}  % Makes := typeset correctly for definitions

%%% Equation numbering
%\numberwithin{equation}{section} 

%%% Annotations
\newcommand{\notate}[1]{\textcolor{red}{\textbf{[#1]}}}

%==============
% Symbols
%==============
\let\oldphi\phi
\let\oldeps\epsilon
\let\oldemptyset\emptyset
\let\emptyset\varnothing

\renewcommand{\phi}{\varphi}
\renewcommand{\epsilon}{\varepsilon}
\newcommand{\eps}{\varepsilon}
\newcommand{\cl}{\mathrm{cl}}
\newcommand{\wto}{\rightharpoonup}
\newcommand{\wsto}{\overset{\ast}{\rightharpoonup}}
\newcommand{\wwto}{\overset{w}{\to}}
\newcommand{\wwsto}{\overset{w*}{\to}}

%==============
% Constants
%==============

% Set constants upright
\newcommand{\cnst}[1]{\mathrm{#1}}  
\newcommand{\econst}{\mathrm{e}}
\newcommand{\rd}{\mathrm{d}}
\newcommand{\dist}{\mathrm{dist}}

\newcommand{\zerovct}{\vct{0}} % Zero vector
\newcommand{\Id}{\mathbf{I}} % Identity matrix
\newcommand{\onemtx}{\bm{1}}
\newcommand{\zeromtx}{\bm{0}}

%==============
% Sets
%==============
\providecommand{\mathbbm}{\mathbb} % In case we don't load bbm

% Reals, complex, naturals, integers, field
\newcommand{\R}{\mathbbm{R}}
\newcommand{\C}{\mathbbm{C}}
\newcommand{\N}{\mathbbm{N}}
\newcommand{\Z}{\mathbbm{Z}}
\newcommand{\F}{\mathbbm{F}}

%==============
% Probability
%==============
\newcommand{\Prob}{\operatorname{\mathbbm{P}}}
\newcommand{\Expect}{\operatorname{\mathbb{E}}}

%==============
% Vectors and matrices 
%==============
\newcommand{\vct}[1]{\mathbold{#1}}
\newcommand{\mtx}[1]{\mathbold{#1}}

%=============
% Operators
%=============
\newcommand{\B}{\mathcal{B}}
\newcommand{\op}[1]{\mathbold{#1}}

 % "macro.tex" must be in the same folder

%>>>>>>> IF NEEDED, ADD A NEW FILE WITH YOUR OWN MACROS

% \input{lecture_01_macro.tex} % Name of supplemental macros should match lecture number

%>>>>>>> LECTURE NUMBER AND TITLE
\title{Clase 03:               % UPDATE LECTURE NUMBER
    Condiciones para evitar el sobreajuste}	% UPDATE TITLE
% TIP:  Use "\\" to break the title into more than one line.

%>>>>>>> DATE OF LECTURE
\date{Enero 21, 2021} % Hard-code lecture date. Don't use "\today"

%>>>>>>> NAME OF SCRIBE(S)
\author{%
  Responsable:&
  Christian Rodríguez Uribe  % >>>>> SCRIBE NAME(S)
}

\begin{document}
\maketitle %  LEAVE HERE
% The command above causes the title to be displayed.

%>>>>> DELETE ALL CONTENT UNTIL "\end{document}"
% This is the body of your document.

Habíamos dicho que existía un $h^*$ que minimizaba el ERM. 
¿Como se ve este punto con respecto a la densidad de donde vienen los puntos y la función etiquetadora?
Hay un cuestión: minimizar el ERM no implica minimizar el error "teórico". La siguiente figura explica un poco lo que sucede:

    \begin{figure}[h!]
      \centering
      \includegraphics[width=1\columnwidth]{digarama1.JPG}
      \caption{{\textsf{Idea general de lo que ocurre entre el minimizador teórico, el minimizador de ERM y el conjunto de clasificadores.}}}\label{fig:bell-curve}
    \end{figure}

En la imagen anterior se muestran los conjuntos de nivel de $L_S(h)$ y de $L_{D, f}(h)$. Se puede ver que los mínimos de ambos errores no corresponden a la misma $h$, entonces lo que se hace es restringir a un conjunto de clasificadores que permita encontrar una $h_H^*$ que 
se aproxime bastante a las $h$'s de los otros errores.  

\subsection*{2.3.1 $H$ es una familia de funciones finita}

La idea de restringir la búsqueda de $h$ sujeta a $H$ es que el $ERM_H$ no sobreajuste. En este sentido, si $H$ es un conjunto grande, entonces S también es un conjunto grande. 

Hipótesis de realizabilidad: $\exists h^* \in H$ tal que $L_{D, f}(h^*) = 0$

Esto implica que, con probabilidad 1, $L_S(h^*) = 0$ y a su vez que $L_S(h_S) = 0$

Para evaluar que nuestro conjunto de entrenamiento es bueno debemos fijarnos en que tanto se parece a D. Entonces hay que revisar que S sea conformado de muestras aleatoria simples.

Como $x^{(i)} \sim D$, entonces $S \sim D^m$.

Veamos que $L_{D.f}(h_S)$ es una variable aleatoria, pues las observaciones S son generadas de manera aleatoria.  

Denotemos por $\delta$ a la probabilidad de obtener una muestra poco representativa de D. Entonces $1 - \delta$ representa la confianza en la precisión de la muestra.

Ahora consideremos $\epsilon$ como la probabilidad de encontrar errores en el etiquetado. Lo que queremos es que $L_{D, f}(h_S) \leq \epsilon$, pues de esta manera aseguramos que nuestro modelo de clasificación va a tener un error menor que la precisión que hay en los datos. Esto es, queremos un clasificador que sea al menos capaz de clasificar con precisión $\epsilon$. Un clasificador que cumpla con esta propiedad lo llamamos que es aproximadamente correcto. 

En resumen, queremos generar una cota para $D^m(\{S:L_{D,f}(h_S)>\epsilon\})$. Es decir, una cota para la probabilidad de $L_{D,f}(h_S)>\epsilon$. Construyámosla.

Sean $H_B$ las hipótesis erróneas y $M$ las muestra engañosas. Es decir, \\$H_B = \{h \in H : L_{D, f}(h) > \epsilon\}$ y $M = \{S: \exists h \in H, L_S(h) = 0\}$

Hipotesis de realizabilidad $\Rightarrow L_S(h_S) = 0$. Si $\exists h \in H_B$ tal que $L_S(h) = 0$ y $S \in M \Rightarrow \{S: L_{D, f}(h_S)>\epsilon\} \subseteq M = \bigcup_{h \in H_B}\{S: L_S(h) = 0\}$. De esta manera tenemos que $D^m(\{S:L_{D,f}(h_S)>\epsilon\}) \leq D^m(M)$. 

Por un lado:

\begin{equation}
    D^m(\bigcup_{h \in H_B}\{S: L_S(h) = 0\}) \leq \sum_{h \in H_B}D^m(\{S: L_S(h) = 0\})
\end{equation}

Por otro lado tenemos que $L_S(h) = 0 \Leftrightarrow h(x^{(i)}) = f(x^{(i)}),  \forall i = 1, ..., m$, entonces $D^m(\{S:L_{S}(h) = 0\}) = \prod_{i = 1}^{m}D(\{x_i: h(x^{(i)}) = f(x^{(i)})\})$. 

Ahora veamos que:

\begin{equation}
    D(\{x_i: h(x^{(i)}) = f(x^{(i)})\}) = 1 - L_{D,f}(h) \leq 1 - \epsilon
\end{equation}

Juntando todo lo anterior, tenemos que:

\begin{equation}
     D^m(\{S:L_{S}(h) = 0\}) \leq (1- \epsilon)^m \leq e^{-\epsilon m}
\end{equation}

Ahora, como tenemos que $|H| < \infty$ y $H_B \subseteq H$, entonces $D^m(\{S:L_{D, f}(h_S) >\epsilon\}) \leq |H_B|e ^{-\epsilon m} \leq |H|e^{-\epsilon m}$

\begin{corollary}
    Sea $H$ una clase finita de hipótesis. Sea $\delta \in (0, 1)$ y $\epsilon > 0$ y sea $m \geq \frac{log(\frac{|H|}{\delta})}{\epsilon}$. Entonces $\forall f$ y distribución $D$, para los cuales se cumple la hipótesis de realizabilidad, tenemos con probabilidad de al menos $1 - \delta$ que toda solución de ERM satisface que $L_{D, f}(h_S) \leq \epsilon$ 
\end{corollary}

Entonces, si $m$ es suficientemente grande, una clase de hipótesis que sea finita será probablemente $(1 - \delta)$ y aproximadamente (hasta un error $\epsilon$) correcta.  


\end{document}
