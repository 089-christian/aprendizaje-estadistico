
\documentclass[11pt,reqno,twoside]{article}
%>>>>>>> RENAME CURRENT FILE TO MATCH LECTURE NUMBER
% E.g., "lecture_01.tex"

%>>>>>>> DO NOT EDIT MACRO FILE

% >>>> DO NOT EDIT THIS FILE
% If you *must* add or change a macro, please email fkoh@caltech.edu

%=================================================
% Basics
%=================================================

\usepackage{fixltx2e} % Makes \( \) equation style robust, among other
                      % things. Must be the first package.


% Makes ligatured fonts searchable and copyable in pdf readers
\usepackage{cmap} % Load before fontenc 

% Always include these font encodings in your document 
% unless you have a very good reason.
\usepackage[T1]{fontenc}
\usepackage[utf8]{inputenc}

\usepackage{verbatim}

%=====================================
% Look & feel
%=====================================

% Allows for manual space setting
\usepackage{setspace}

%=============
% Fonts
%=============

\usepackage{lmodern} % Improved version of computer modern
\usepackage[scale=0.88]{tgheros} % Helvetica clone for sans serif font


\newcommand\hmmax{2} % Default is 3.
\newcommand\bmmax{2} % Default is 4.

\usepackage{bm} % boldmath must be called after the package
\providecommand{\mathbold}[1]{\bm{#1}}

%=============
% AMS Packages and fonts
%=============
\usepackage{amsmath,amsbsy,amsgen,amscd,amsthm,amsfonts,amssymb} 

%=============
% Margins and paper size
%=============
\usepackage[centering,top=1.5in,bottom=1.2in,left=1.4in,right=1.4in]{geometry}

%=============
% Title setup
%=============
\usepackage{titling}
%\usepackage{nopageno}
\setlength{\droptitle}{-7.5em}

\pretitle{\noindent\rule{0.85\linewidth}{0.2mm}\par%
  \begin{raggedright}\LARGE\sffamily}
\posttitle{\par\end{raggedright}%
\noindent%
\rule{0.85\linewidth}{0.5mm}\par}

\preauthor{\noindent\vspace{0.5em}%
  \sffamily\begin{tabular}[t]{ll}}
  \postauthor{\end{tabular}\par\thispagestyle{plain}}

\predate{\noindent%
  \small\sffamily\itshape\begin{tabular}[t]{l}%
    ACM 256, Winter 2019 \\ %
    Dr.\ Franca Hoffmann \\ %
  }
  \postdate{
  \end{tabular}\par}

%=============
% Section headings
%=============
\usepackage[sf,bf,compact]{titlesec}

%=============
% Tables and lists
%=============
\usepackage{booktabs,longtable,tabu} % Nice tables
\setlength{\tabulinesep}{1mm}
\usepackage[font=small,margin=10pt,labelfont={sf,bf},labelsep={space}]{caption}


\usepackage{enumitem}
\setitemize{itemsep=0pt} 
\setenumerate{itemsep=0pt}
\setlist{labelindent=\parindent,%  % Recommended by enumitem package
  font=\sffamily}


%=============
% Hyperlink colors
%=============
\usepackage[usenames,dvipsnames]{xcolor}
\definecolor{dark-gray}{gray}{0.3}
\definecolor{dkgray}{rgb}{.4,.4,.4}
\definecolor{dkblue}{rgb}{0,0,.5}
\definecolor{medblue}{rgb}{0,0,.75}
\definecolor{rust}{rgb}{0.5,0.1,0.1}

\usepackage{url}
\usepackage[colorlinks=true]{hyperref}
\hypersetup{linkcolor=dkblue}    
\hypersetup{citecolor=rust}      
\hypersetup{urlcolor=rust}     

%=============
% Microtype
%=============
\usepackage[final]{microtype} 

%=============
% Theorems, etc.
%=============
\newtheoremstyle{myThm} % name
    {\topsep}                    % Space above
    {\topsep}                    % Space below
    {\itshape}                   % Body font
    {}                           % Indent amount
    {\sffamily\bfseries}                   % Theorem head font
    {.}                          % Punctuation after theorem head
    {.5em}                       % Space after theorem head
    {}  % Theorem head spec (can be left empty, meaning ‘normal’)

\newtheoremstyle{myRem} % name
    {\topsep}                    % Space above
    {\topsep}                    % Space below
    {}                   % Body font
    {}                           % Indent amount
    {\sffamily}                   % Theorem head font
    {.}                          % Punctuation after theorem head
    {.5em}                       % Space after theorem head
    {}  % Theorem head spec (can be left empty, meaning ‘normal’)

\newtheoremstyle{myDef} % name
    {\topsep}                    % Space above
    {\topsep}                    % Space below
    {}                   % Body font
    {}                           % Indent amount
    {\sffamily\bfseries}                   % Theorem head font
    {.}                          % Punctuation after theorem head
    {.5em}                       % Space after theorem head
    {}  % Theorem head spec (can be left empty, meaning ‘normal’)

\theoremstyle{myThm}
\newtheorem{theorem}{Theorem}[section]
\newtheorem{lemma}[theorem]{Lemma}
\newtheorem{proposition}[theorem]{Proposition}
\newtheorem{corollary}[theorem]{Corollary}
\newtheorem{fact}[theorem]{Fact}

\theoremstyle{myRem}
\newtheorem{remark}[theorem]{Remark}

\theoremstyle{myDef}
\newtheorem{definition}[theorem]{Definition}
\newtheorem{example}[theorem]{Example}

%=====================
% Header
%=====================
\usepackage{fancyhdr}
\usepackage{nopageno} % Gets rid of page number at the bottom
\fancyhf{} % Clear header style
\renewcommand{\headrulewidth}{0.5pt} % remove the header rule
\pagestyle{fancy}
\fancyhead[LE,RO]{\textsf{\small \thepage}}

\setlength{\headheight}{14pt}
%=====================
% Fix delimiters
%=====================

% Fixes \left and \right spacing issues. See discussion at
% http://tex.stackexchange.com/questions/2607/spacing-around-left-and-right
\let\originalleft\left
\let\originalright\right
\renewcommand{\left}{\mathopen{}\mathclose\bgroup\originalleft}
\renewcommand{\right}{\aftergroup\egroup\originalright}

%=================================================
% Math macros
%=================================================

%=============
% Generalities
%=============
\usepackage{mathtools}
\mathtoolsset{centercolon}  % Makes := typeset correctly for definitions

%%% Equation numbering
%\numberwithin{equation}{section} 

%%% Annotations
\newcommand{\notate}[1]{\textcolor{red}{\textbf{[#1]}}}

%==============
% Symbols
%==============
\let\oldphi\phi
\let\oldeps\epsilon
\let\oldemptyset\emptyset
\let\emptyset\varnothing

\renewcommand{\phi}{\varphi}
\renewcommand{\epsilon}{\varepsilon}
\newcommand{\eps}{\varepsilon}
\newcommand{\cl}{\mathrm{cl}}
\newcommand{\wto}{\rightharpoonup}
\newcommand{\wsto}{\overset{\ast}{\rightharpoonup}}
\newcommand{\wwto}{\overset{w}{\to}}
\newcommand{\wwsto}{\overset{w*}{\to}}

%==============
% Constants
%==============

% Set constants upright
\newcommand{\cnst}[1]{\mathrm{#1}}  
\newcommand{\econst}{\mathrm{e}}
\newcommand{\rd}{\mathrm{d}}
\newcommand{\dist}{\mathrm{dist}}

\newcommand{\zerovct}{\vct{0}} % Zero vector
\newcommand{\Id}{\mathbf{I}} % Identity matrix
\newcommand{\onemtx}{\bm{1}}
\newcommand{\zeromtx}{\bm{0}}

%==============
% Sets
%==============
\providecommand{\mathbbm}{\mathbb} % In case we don't load bbm

% Reals, complex, naturals, integers, field
\newcommand{\R}{\mathbbm{R}}
\newcommand{\C}{\mathbbm{C}}
\newcommand{\N}{\mathbbm{N}}
\newcommand{\Z}{\mathbbm{Z}}
\newcommand{\F}{\mathbbm{F}}

%==============
% Probability
%==============
\newcommand{\Prob}{\operatorname{\mathbbm{P}}}
\newcommand{\Expect}{\operatorname{\mathbb{E}}}

%==============
% Vectors and matrices 
%==============
\newcommand{\vct}[1]{\mathbold{#1}}
\newcommand{\mtx}[1]{\mathbold{#1}}

%=============
% Operators
%=============
\newcommand{\B}{\mathcal{B}}
\newcommand{\op}[1]{\mathbold{#1}}

 % "macro.tex" must be in the same folder

%>>>>>>> IF NEEDED, ADD A NEW FILE WITH YOUR OWN MACROS

% \input{lecture_01_macro.tex} % Name of supplemental macros should match lecture number

%>>>>>>> LECTURE NUMBER AND TITLE
\title{Clase 11:               % UPDATE LECTURE NUMBER
    Aprendizaje Estadístico}	% UPDATE TITLE
% TIP:  Use "\\" to break the title into more than one line.

%>>>>>>> DATE OF LECTURE
\date{Febrero 23, 2021} % Hard-code lecture date. Don't use "\today"

%>>>>>>> NAME OF SCRIBE(S)
\author{%
  Responsable: José Pablo Sánchez
  % >>>>> SCRIBE NAME(S)
}

\begin{document}
\maketitle %  LEAVE HERE
% The command above causes the title to be displayed.

%>>>>> DELETE ALL CONTENT UNTIL "\end{document}"
% This is the body of your document.
\section{Clase 11: Aprendizaje no uniforme (NUL)}
\label{sec:introduction}
Vimos que en el modelo PAC establecíamos una relacion entre el tamaño de muestra (m) y los parámetros ($\epsilon$,$\delta$). Estos parámetros son uniformes con respecto a f y D.
\\
\hspace*{20mm}  $\Rightarrow{} \text{las clases son limitadas} (VCdim(\mathcal{H}) < \infty$
\\
- Ahora buscamos cómo relajar la noción de aprendizaje
\\
- NUL $\longrightarrow$ incorpora una hipótesis $(h \in \mathcal{H})$ contra la que estamos comparando. Esto relaja PAC agnóstico.
\\
- Caracterizar: una unión numerable de posibles clases donde cada elemento es uniforme.
\\
- Esto da lugar al paradigma de minimización de riesgo estructural (SRM)

\subsection{Capacidad de aprendizaje no uniforme (NUL)}
\textbf{Definición: }Una Hipótesis $(h)$ es ($\epsilon,\delta$)-competitiva con respecto a ${h}'$ si con probabilidad $\geq 1 - \delta$ se cumple:
\begin{equation}
L_{D}(h) \leq L_{D}({h}') + \epsilon
\end{equation}
\newline \textbf{Definición: }Una clase $\mathcal{H}$ es aprendible no uniformemente (NUL) si existe un algoritmo de aprendizaje, A, y una función $M_{H}^{NUL}: (0,1)^2\times H \longrightarrow \mathbb{N}$ tal que $(\epsilon,\delta)\in(0,1)^2$ y $h \in \mathcal{H}$.
\newline Si $m \geq M_{H}^{NUL}(\epsilon,\delta,h)$ entonces $\forall D$ con probabilidad $\geq 1 - \delta$ bajo $S \sim D^m$ tenemos que
\begin{equation}
L_{D}(A(S)) \leq L_{D}(h) + \epsilon
\end{equation}

\subsection{Caracterización de NUL}
\begin{theorem}
\newline Una clase $\mathcal{H}$ de clasificadores binarios es NUL sí y sólo sí es una unión numerable de clases PAC agnósticas.
\end{theorem}
\begin{theorem}
\newline Sea $\mathcal{H}$ una clase tal que $\mathcal{H} = \bigcup\limits_{n \in \mathbb{N}}\mathcal{H}_{n}$ donde cada $\mathcal{H}_{n}$ es uniforme. Entonces $\mathcal{H}$ es NUL.
\end{theorem}

\underline{Ejemplo}
\newline Sean $\mathcal{H}_{n} = \{\text{clasificadores polinomios de grando = n}\}$, es decir $h \in \mathcal{H}_{n},$ entonces $h(x) = signo(P_n(x))$
\newline Sea $\mathcal{H} = \bigcup\limits_{n = 1}\mathcal{H}_{n} = \{\text{todos los polinomios posibles} \in \mathbb{R}\}$, luego es fácil ver que $VCdim(\mathcal{H}) = \infty$ y que $VCdim(\mathcal{H}_{n}) \leq n + 1$
\newline \hspace*{20mm} $\therefore \mathcal{H}$ no será PAC agnóstico, pero, por los teoremas anteriores, $\mathcal{H}$ es NUL

\subsection{Minimización de Riesgo Estructural (SRM)}
\newline Si representamos nuestro espacio de posibles hipótesis $\mathcal{H} = \bigcup\limits_{n \in \mathbb{N}}\mathcal{H}_{n}$ tendremos que asignar un $W_n$ (peso) para cada $\mathcal{H}_n$
\\
\newline \textbf{Definición:} Sea $\mathcal{H} = \bigcup\limits_{n \in \mathbb{N}}\mathcal{H}_{n}$ con cada $\mathcal{H}_{n}$ uniforme con $m_{H}^{UC}(\epsilon,\delta)$ y definimos $\varepsilon_n:\mathbb{N} \times (0,1) \longrightarrow (0,1)$ como:
\begin{equation}
\varepsilon_n(m,\delta) = min.\{\epsilon \in (0,1):m_{H}^{UC}(\epsilon,\delta) \leq m\}
\end{equation}

\newline \underline{Nota:} nos fijamos en una cota mínima posible usando n obervaciones.
\\
Si utilizamos la definición de convergencia uniforme y la definición de $\varepsilon_n$ tenemos que $\forall (\epsilon,\delta)$ con probabilidad $\geq 1 - \delta$ bajo $S \sim D^m$ se satisface que
\begin{equation}
    \forall h \in \mathcal{H}_n,\text{ }\left | L_{D}(h) - L_{S}(h)\right |\leq\varepsilon_n(m,\delta)
\end{equation}
\\
Tomemos ahora $W_n$ tal que $\sum W_n \leq 1$
\\
Si tenemos N posibles candidatos $\mathcal{H}_n$ podríamos considerar cada familia con el mismo peso $W_n = \frac{1}{N}$, más esto no es posible en el caso infinito

\begin{theorem}
 Sea $W:\mathbb{N} \longrightarrow [0,1]$ tal que $ \sum_{n=1}^{\infty} W(n) \leq 1$.
 \\Sea $\mathcal{H} = \bigcup\limits_{n = 1}^{\infty}\mathcal{H}_{n}$ con cada $\mathcal{H}_{n}$ uniforme con $m_{H_n}^{UC}$.
\\ Sea $\varepsilon_n$ como arriba, entonces $\forall \delta\in(0,1)$ y D con probabilidad $\geq 1 - \delta$ sobre
$S \sim D^m$ se satisface de manera simultanea, es decir $\forall n \in \mathbb{N}$ y $h \in \mathacal{H}_n$, la desigualdad \begin{equation}
    \left | L_{D}(h) - L_{S}(h)\right |\leq\varepsilon_n(m,W_n\delta)
\end{equation}
\newline \hspace*{10mm} $\therefore \forall\delta \in (0,1)$ y $D$ con probabilidad $\geq 1 - \delta$ se cumplirá $\forall h \in \mathcal{H}$
\begin{equation}
     L_{D}(h) \leq L_{S}(h) + \min\varepsilon_n(m,W_n\delta)
\end{equation}
\newline Notemos que $n = n(h) = \min\{n: h \in \mathcal{H}_n\}$
\end{theorem}
\begin{theorem}
 Sea $\mathcal{H} = \bigcup\limits_{n = 1}^{\infty}\mathcal{H}_{n}$ con cada $\mathcal{H}_{n}$ uniforme con $m_{H_n}^{UC}$.
 \\Sea $W:\mathbb{N} \longrightarrow [0,1]$ tal que $W(n) = \frac{6}{n^2\pi^2}$.
 \\Entonces $\mathcal{H}$ es NUL usando el SRM con
 \begin{equation}
     m_{H}^{NUL}(\epsilon,\delta,h) \leq m_{H}^{UC}(\frac{\epsilon}{2},W(n)\delta).
 \end{equation}

\end{theorem}

\section{Resumen}
\newline - Nuestra cota en el error de generalización se basa en evidencia empírica (error de entrenamiento)
\newline - No podemos establecer un tamaño de muestra suficiente, y dependerá del mejor candidato $h \in \mathcal{H} \implies$ la calidad de nuestra respuesta depende de nuestras preferencias.
\newline - Nos Ayudará a seleccionar modelos cuando nuestra información previa es incompleta



\end{document}
