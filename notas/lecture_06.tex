\documentclass[11pt,reqno,twoside]{article}
%>>>>>>> RENAME CURRENT FILE TO MATCH LECTURE NUMBER
% E.g., "lecture_01.tex"

%>>>>>>> DO NOT EDIT MACRO FILE

% >>>> DO NOT EDIT THIS FILE
% If you *must* add or change a macro, please email fkoh@caltech.edu

%=================================================
% Basics
%=================================================

\usepackage{fixltx2e} % Makes \( \) equation style robust, among other
                      % things. Must be the first package.


% Makes ligatured fonts searchable and copyable in pdf readers
\usepackage{cmap} % Load before fontenc 

% Always include these font encodings in your document 
% unless you have a very good reason.
\usepackage[T1]{fontenc}
\usepackage[utf8]{inputenc}

\usepackage{verbatim}

%=====================================
% Look & feel
%=====================================

% Allows for manual space setting
\usepackage{setspace}

%=============
% Fonts
%=============

\usepackage{lmodern} % Improved version of computer modern
\usepackage[scale=0.88]{tgheros} % Helvetica clone for sans serif font


\newcommand\hmmax{2} % Default is 3.
\newcommand\bmmax{2} % Default is 4.

\usepackage{bm} % boldmath must be called after the package
\providecommand{\mathbold}[1]{\bm{#1}}

%=============
% AMS Packages and fonts
%=============
\usepackage{amsmath,amsbsy,amsgen,amscd,amsthm,amsfonts,amssymb} 

%=============
% Margins and paper size
%=============
\usepackage[centering,top=1.5in,bottom=1.2in,left=1.4in,right=1.4in]{geometry}

%=============
% Title setup
%=============
\usepackage{titling}
%\usepackage{nopageno}
\setlength{\droptitle}{-7.5em}

\pretitle{\noindent\rule{0.85\linewidth}{0.2mm}\par%
  \begin{raggedright}\LARGE\sffamily}
\posttitle{\par\end{raggedright}%
\noindent%
\rule{0.85\linewidth}{0.5mm}\par}

\preauthor{\noindent\vspace{0.5em}%
  \sffamily\begin{tabular}[t]{ll}}
  \postauthor{\end{tabular}\par\thispagestyle{plain}}

\predate{\noindent%
  \small\sffamily\itshape\begin{tabular}[t]{l}%
    ACM 256, Winter 2019 \\ %
    Dr.\ Franca Hoffmann \\ %
  }
  \postdate{
  \end{tabular}\par}

%=============
% Section headings
%=============
\usepackage[sf,bf,compact]{titlesec}

%=============
% Tables and lists
%=============
\usepackage{booktabs,longtable,tabu} % Nice tables
\setlength{\tabulinesep}{1mm}
\usepackage[font=small,margin=10pt,labelfont={sf,bf},labelsep={space}]{caption}


\usepackage{enumitem}
\setitemize{itemsep=0pt} 
\setenumerate{itemsep=0pt}
\setlist{labelindent=\parindent,%  % Recommended by enumitem package
  font=\sffamily}


%=============
% Hyperlink colors
%=============
\usepackage[usenames,dvipsnames]{xcolor}
\definecolor{dark-gray}{gray}{0.3}
\definecolor{dkgray}{rgb}{.4,.4,.4}
\definecolor{dkblue}{rgb}{0,0,.5}
\definecolor{medblue}{rgb}{0,0,.75}
\definecolor{rust}{rgb}{0.5,0.1,0.1}

\usepackage{url}
\usepackage[colorlinks=true]{hyperref}
\hypersetup{linkcolor=dkblue}    
\hypersetup{citecolor=rust}      
\hypersetup{urlcolor=rust}     

%=============
% Microtype
%=============
\usepackage[final]{microtype} 

%=============
% Theorems, etc.
%=============
\newtheoremstyle{myThm} % name
    {\topsep}                    % Space above
    {\topsep}                    % Space below
    {\itshape}                   % Body font
    {}                           % Indent amount
    {\sffamily\bfseries}                   % Theorem head font
    {.}                          % Punctuation after theorem head
    {.5em}                       % Space after theorem head
    {}  % Theorem head spec (can be left empty, meaning ‘normal’)

\newtheoremstyle{myRem} % name
    {\topsep}                    % Space above
    {\topsep}                    % Space below
    {}                   % Body font
    {}                           % Indent amount
    {\sffamily}                   % Theorem head font
    {.}                          % Punctuation after theorem head
    {.5em}                       % Space after theorem head
    {}  % Theorem head spec (can be left empty, meaning ‘normal’)

\newtheoremstyle{myDef} % name
    {\topsep}                    % Space above
    {\topsep}                    % Space below
    {}                   % Body font
    {}                           % Indent amount
    {\sffamily\bfseries}                   % Theorem head font
    {.}                          % Punctuation after theorem head
    {.5em}                       % Space after theorem head
    {}  % Theorem head spec (can be left empty, meaning ‘normal’)

\theoremstyle{myThm}
\newtheorem{theorem}{Theorem}[section]
\newtheorem{lemma}[theorem]{Lemma}
\newtheorem{proposition}[theorem]{Proposition}
\newtheorem{corollary}[theorem]{Corollary}
\newtheorem{fact}[theorem]{Fact}

\theoremstyle{myRem}
\newtheorem{remark}[theorem]{Remark}

\theoremstyle{myDef}
\newtheorem{definition}[theorem]{Definition}
\newtheorem{example}[theorem]{Example}

%=====================
% Header
%=====================
\usepackage{fancyhdr}
\usepackage{nopageno} % Gets rid of page number at the bottom
\fancyhf{} % Clear header style
\renewcommand{\headrulewidth}{0.5pt} % remove the header rule
\pagestyle{fancy}
\fancyhead[LE,RO]{\textsf{\small \thepage}}

\setlength{\headheight}{14pt}
%=====================
% Fix delimiters
%=====================

% Fixes \left and \right spacing issues. See discussion at
% http://tex.stackexchange.com/questions/2607/spacing-around-left-and-right
\let\originalleft\left
\let\originalright\right
\renewcommand{\left}{\mathopen{}\mathclose\bgroup\originalleft}
\renewcommand{\right}{\aftergroup\egroup\originalright}

%=================================================
% Math macros
%=================================================

%=============
% Generalities
%=============
\usepackage{mathtools}
\mathtoolsset{centercolon}  % Makes := typeset correctly for definitions

%%% Equation numbering
%\numberwithin{equation}{section} 

%%% Annotations
\newcommand{\notate}[1]{\textcolor{red}{\textbf{[#1]}}}

%==============
% Symbols
%==============
\let\oldphi\phi
\let\oldeps\epsilon
\let\oldemptyset\emptyset
\let\emptyset\varnothing

\renewcommand{\phi}{\varphi}
\renewcommand{\epsilon}{\varepsilon}
\newcommand{\eps}{\varepsilon}
\newcommand{\cl}{\mathrm{cl}}
\newcommand{\wto}{\rightharpoonup}
\newcommand{\wsto}{\overset{\ast}{\rightharpoonup}}
\newcommand{\wwto}{\overset{w}{\to}}
\newcommand{\wwsto}{\overset{w*}{\to}}

%==============
% Constants
%==============

% Set constants upright
\newcommand{\cnst}[1]{\mathrm{#1}}  
\newcommand{\econst}{\mathrm{e}}
\newcommand{\rd}{\mathrm{d}}
\newcommand{\dist}{\mathrm{dist}}

\newcommand{\zerovct}{\vct{0}} % Zero vector
\newcommand{\Id}{\mathbf{I}} % Identity matrix
\newcommand{\onemtx}{\bm{1}}
\newcommand{\zeromtx}{\bm{0}}

%==============
% Sets
%==============
\providecommand{\mathbbm}{\mathbb} % In case we don't load bbm

% Reals, complex, naturals, integers, field
\newcommand{\R}{\mathbbm{R}}
\newcommand{\C}{\mathbbm{C}}
\newcommand{\N}{\mathbbm{N}}
\newcommand{\Z}{\mathbbm{Z}}
\newcommand{\F}{\mathbbm{F}}

%==============
% Probability
%==============
\newcommand{\Prob}{\operatorname{\mathbbm{P}}}
\newcommand{\Expect}{\operatorname{\mathbb{E}}}

%==============
% Vectors and matrices 
%==============
\newcommand{\vct}[1]{\mathbold{#1}}
\newcommand{\mtx}[1]{\mathbold{#1}}

%=============
% Operators
%=============
\newcommand{\B}{\mathcal{B}}
\newcommand{\op}[1]{\mathbold{#1}}

 % "macro.tex" must be in the same folder

%>>>>>>> IF NEEDED, ADD A NEW FILE WITH YOUR OWN MACROS

% \input{lecture_01_macro.tex} % Name of supplemental macros should match lecture number

%>>>>>>> LECTURE NUMBER AND TITLE
\title{Clase 6:               % UPDATE LECTURE NUMBER
    Modelos lineales}	% UPDATE TITLE
% TIP:  Use "\\" to break the title into more than one line.

%>>>>>>> DATE OF LECTURE
\date{Febrero 2, 2021} % Hard-code lecture date. Don't use "\today"

%>>>>>>> NAME OF SCRIBE(S)
\author{%
  Responsable:&
  Carlos Enrique Lezama Jacinto  % >>>>> SCRIBE NAME(S)
}

\begin{document}
\maketitle

\section{Predictores lineales}

En primer lugar, nos enfocaremos en predictores lineales\footnote{Inicialmente, utilizaremos la \emph{minimización de riesgo empírico} (ERM, por sus siglas en inglés) como método de aprendizaje preferido.} puesto que son fáciles de interpretar y se ajustan razonablemente bien a los datos de entrenamiento. Además, son muy intuitivos y presentan las bases para modelos más complejos.

Algunos de los modelos y algortimos que estudiaremos en las próximas clases son:

\vspace{10px}

\begin{minipage}[c]{\linewidth}
	\begin{center}
		 \begin{tabu}{ccc}
			\toprule
		    \textbf{Modelos} & \textbf{Algoritmos} & \textbf{Tarea} \\
            \midrule
		    Semiespacios & Programación lineal (LP), perceptrones & Clasificación \\
		    Regresión lineal & Mínimos cuadrados & Regresión \\
		    Regresión logística & Métodos iterativos & Clasificación \\
            \bottomrule
        \end{tabu}
	\end{center}
\end{minipage}

\vspace{10px}

\begin{definition}[Clase de transformaciones afines]
	Definimos la clase de transformaciones afines como:
	\[
		L_p = \{ h_{\vct{w}, b}\ :\ \vct{w} \in \R^p,\ b \in \R \}
	,\]
	donde
	\[
		h_{\vct{w}, b} (\vct{x}) = \langle \vct{w}, \vct{x} \rangle + b = \left( \sum_{i = 1}^{p} w_i x_i \right) + b
	.\]
\end{definition}

Las clases que veremos a continuación son composiciones de una función $\phi : \R \to \mathcal{Y}$ con $L_p$. En un caso de \emph{clasificación binaria}, podemos proponer:

\begin{center}
\begin{equation*}
	\phi\left( h(x) \right) =
	\begin{cases}
		1, & h(x) > 0 \\
		0, & \text{en otros casos}
	\end{cases}
\end{equation*}
\end{center}

Por otro lado, para problemas de \emph{regresión}, nuestra $\phi$ fácilmente puede ser la función identidad, i.e. $\phi\left( h\left( x \right) \right) = h\left( x \right)$.

\subsection{Modelo de semiespacios}

\begin{definition}[Clase de semiespacios]
	La clase de semiespacios, diseñada para problemas de clasificación binaria con $\mathcal{X} = \R^p$ y  $\mathcal{Y} = \{-1, +1\}$, la definimos como:
	 \[
		 HS_p = \text{sgn} \circ L_p = \{\text{sgn} \left( \langle \vct{w}, \vct{x} \rangle \right) : \vct{w} \in \R^p \}
	.\]
\end{definition}

\newpage

Dada la definición anterior, podemos considerar las siguientes tres situaciones:

\begin{enumerate}
	\item\label{itm:separable} Casos separables al asumir que se cumple la \emph{hipótesis de realizabilidad}.
	\item Casos no-separables en un sentido agnóstico.
	\item[\textcolor{red}{3.}] Cuando una función lineal no es suficiente y necesitamos una transformación no lineal.
\end{enumerate}

\subsubsection{Programación lineal}

Los programas lineales son problemas que pueden expresarse como la maximización de una función lineal sujeta a restricciones lineales. Es decir,

\begin{center}
	\begin{equation*}
		\begin{aligned}
			\max_{\vct{w} \in \R_p} \qquad & \langle \vct{u}, \vct{w} \rangle \\
			\text{sujeto a} \qquad & A\vct{w} \ge \vct{v}
		\end{aligned}
	\end{equation*}
\end{center}

\begin{remark}
	Sea $\displaystyle S = \left\{ \left( \vct{x}^{(i)}, y^{(i)} \right) \right\}_{i = 1}^{m}$ un conjunto de entrenamiento. Al asumir el caso \ref{itm:separable}, existe $\vct{w} \in \R^p$ tal que
	\[
		y_i \langle \vct{w}, \vct{x}^{(i)} \rangle \ge 1, \qquad \forall i = 1, \ldots, m
	,\] y $\vct{w}$ es un predictor ERM.
\end{remark}

Por la observación anterior, si definimos $A \in \R^{m \times p}$ tal que $A_{i,j} = y^{(i)} x^{(i)}_j$, entonces podemos escribir $A\vct{w} \ge \vct{v}$.

\subsubsection{Algoritmo de perceptrones}

Este algortimo iterativo construye una secuencia de vectores $\vct{w}^{(1)}, \vct{w}^{(2)},\ldots$. Inicialmente, establecemos $\vct{w}^{(1)} = \bar{0}$ (vector de ceros). En la iteración $t$, el Perceptrón encuentra un ejemplo $i$ mal etiquetado por $\vct{w}^{(t)}$, es decir, $y^{(i)} \langle \vct{w}^{(t)}, \vct{x}^{(i)} \rangle < 0$. Entonces, el Perceptrón actualiza $\vct{w}^{(t + 1)} = \vct{w}^{(t)} + y^{(i)} \vct{x}^{(i)}$. Recordemos que nuestro objetivo es tener $y^{(i)} \langle \vct{w}, \vct{x}^{(i)} \rangle > 0$ para toda $i$. Nótese que
\[
	y^{(i)} \langle \vct{w}^{(t+1)}, \vct{x}^{(i)} \rangle = y^{(i)} \langle \vct{w}^{(t)} + y^{(i)}\vct{x}^{(i)}, \vct{x}^{(i)} \rangle = y^{(i)} \langle \vct{w}^{(t)}, \vct{x}^{(i)} \rangle + \|\vct{w}^{(i)}\|^2
.\]

\begin{theorem}
	Al asumir $\left\{\left(x^{(i)}, y^{(i)}\right)\right\}_{i=1}^m$ separables, y sean $B = \min \left\{ \| \vct{w} \| : y^{(i)} \langle \vct{w}, \vct{x}^{(i)} \rangle \ge 1 \right\}$, y $\displaystyle R = \max_{i} \left\{ \| \vct{x}^{(i)} \| \right\}$. Entonces, el algortimo de perceptrones se detiene a lo más $\left( RB \right)^2 $ iteraciones después y devuelve $\vct{w}^{(t)}$ tal que $y^{(i)} \langle \vct{w}^{(t)}, \vct{x}^{(i)} \rangle > 0$.
\end{theorem}

\begin{proof}
	Bastante larga para estas notas.
\end{proof}

\end{document}
